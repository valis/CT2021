\documentclass[draft]{article}
\usepackage[russian]{babel}
\usepackage[utf8]{inputenc}
\usepackage{cmap}
\usepackage{amsfonts}
\usepackage{amssymb}
\usepackage{amsmath}
\usepackage[all]{xy}

\newcommand{\cat}[1]{\mathbf{#1}}
\renewcommand{\C}{\cat{C}}
\newcommand{\D}{\cat{D}}
\newcommand{\Set}{\cat{Set}}
\newcommand{\FinSet}{\cat{FinSet}}
\newcommand{\Grp}{\cat{Grp}}
\newcommand{\Mon}{\cat{Mon}}
\newcommand{\CMon}{\cat{CMon}}
\newcommand{\Ab}{\cat{Ab}}
\newcommand{\Mat}{\cat{Mat}}
\newcommand{\Num}{\cat{Num}}
\newcommand{\fs}[1]{\mathrm{#1}}
\newcommand{\id}{\fs{id}}

\newenvironment{tolerant}[1]{\par\tolerance=#1\relax}{\par}

\begin{document}

\title{Задания}
\maketitle

\begin{enumerate}

\item На второй лекции мы видели, что морфизм групп является мономорфизмом тогда и только тогда, когда мономорфизмом является соответствующая ему функция на множествах.
Сейчас мы можем обобщить это утверждение.
Забывающий функтор $U : \Grp \to \Set$ является правым сопряженным и строгим.
Для любого функтора, удовлетворяющего этим двум условиям, можно доказать аналогичное утверждение.

Пусть $U : \C \to \D$ -- некоторый функтор.
Докажите следующие утверждения:
\begin{enumerate}
\item Если $U$ является правым сопряженным, то он сохраняет мономорфизмы.
\item Если $U$ является строгим, то обратное верно, то есть если $U(f)$ -- мономорфизм, то $f$ также является мономорфизмом.
\end{enumerate}

\item Докажите, что у забывающего функтора $U : \cat{Cat} \to \cat{Graph}$, сконструированного в 5 ДЗ, существует левый сопряженный.

\item Докажите, что левый сопряженный к некоторому функтору $U$ уникален с точностью до изоморфизма, то есть если $F \dashv U$ и $F' \dashv U$, то $F \simeq F'$.

\item Есть ли у забывающего функтора $U : \Grp \to \Set$ правый сопряженный? Докажите это.

\item Есть ли у забывающего функтора $U : \Grp \to \Mon$ правый сопряженный? Докажите это.

\item Пусть $\cat{rGraph}$ -- категорий рефлексивных графов.
Объекты этой категории -- это графы, в которых для каждой вершины $x$ выбрана петля $id_x$ в этой вершине.
Морфизмы -- морфизмы графов, сохраняющие тождественные петли.

Категория графов в данном упражнении не будет работать, но вместо $\cat{rGraph}$ можно взять категорию малых группоидов или категорию малых категорий; решение при этом не изменится.

Докажите, что у функтора $\Gamma : \cat{rGraph} \to \Set$, сопоставляющего каждому рефлексивному графу множество его вершин, существует правый сопряженный $C : \Set \to \cat{rGraph}$ и левый сопряженный $D : \Set \to \cat{rGraph}$,
и у $D$ существует левый сопряженный $\Pi_0 : \cat{rGraph} \to \Set$.
Таким образом, мы получаем следующую цепочку сопряженных функторов:
\[ \Pi_0 \dashv D \dashv \Gamma \dashv C \]

\item Докажите, что категории $\cat{Fam}_I$ и $\Set/I$ эквивалентны.

\item Пусть $\C$ -- декартовая категория.
Если $A$ -- объект $\C$, то мы можем определить функтор $A^* : \C \to \C/A$ как $A^*(B) = (A \times B, \pi_1)$ и $A^*(f) = \id_A \times f$.
\begin{itemize}
\item Докажите, что у $A^*$ есть левый сопряженный.
\item Докажите, что если $\C$ декартово замкнута и в $\C$ есть уравнители, то у $A^*$ есть правый сопряженный.
\end{itemize}
\end{enumerate}

\end{document}
