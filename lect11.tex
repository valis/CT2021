\documentclass{beamer}

\usepackage{etex}
\usepackage[english,russian]{babel}
\usepackage[utf8]{inputenc}
\usepackage[all]{xy}
\usepackage{ifthen}
\usepackage{xargs}
\usepackage{stmaryrd}
\usepackage{bussproofs}
\usepackage{turnstile}

\usetheme{Szeged}
% \usetheme{Montpellier}
% \usetheme{Malmoe}
% \usetheme{Berkeley}
% \usetheme{Hannover}
\usecolortheme{beaver}

\renewcommand{\turnstile}[6][s]
    {\ifthenelse{\equal{#1}{d}}
        {\sbox{\first}{$\displaystyle{#4}$}
        \sbox{\second}{$\displaystyle{#5}$}}{}
    \ifthenelse{\equal{#1}{t}}
        {\sbox{\first}{$\textstyle{#4}$}
        \sbox{\second}{$\textstyle{#5}$}}{}
    \ifthenelse{\equal{#1}{s}}
        {\sbox{\first}{$\scriptstyle{#4}$}
        \sbox{\second}{$\scriptstyle{#5}$}}{}
    \ifthenelse{\equal{#1}{ss}}
        {\sbox{\first}{$\scriptscriptstyle{#4}$}
        \sbox{\second}{$\scriptscriptstyle{#5}$}}{}
    \setlength{\dashthickness}{0.111ex}
    \setlength{\ddashthickness}{0.35ex}
    \setlength{\leasturnstilewidth}{2em}
    \setlength{\extrawidth}{0.2em}
    \ifthenelse{%
      \equal{#3}{n}}{\setlength{\tinyverdistance}{0ex}}{}
    \ifthenelse{%
      \equal{#3}{s}}{\setlength{\tinyverdistance}{0.5\dashthickness}}{}
    \ifthenelse{%
      \equal{#3}{d}}{\setlength{\tinyverdistance}{0.5\ddashthickness}
        \addtolength{\tinyverdistance}{\dashthickness}}{}
    \ifthenelse{%
      \equal{#3}{t}}{\setlength{\tinyverdistance}{1.5\dashthickness}
        \addtolength{\tinyverdistance}{\ddashthickness}}{}
        \setlength{\verdistance}{0.4ex}
        \settoheight{\lengthvar}{\usebox{\first}}
        \setlength{\raisedown}{-\lengthvar}
        \addtolength{\raisedown}{-\tinyverdistance}
        \addtolength{\raisedown}{-\verdistance}
        \settodepth{\raiseup}{\usebox{\second}}
        \addtolength{\raiseup}{\tinyverdistance}
        \addtolength{\raiseup}{\verdistance}
        \setlength{\lift}{0.8ex}
        \settowidth{\firstwidth}{\usebox{\first}}
        \settowidth{\secondwidth}{\usebox{\second}}
        \ifthenelse{\lengthtest{\firstwidth = 0ex}
            \and
            \lengthtest{\secondwidth = 0ex}}
                {\setlength{\turnstilewidth}{\leasturnstilewidth}}
                {\setlength{\turnstilewidth}{2\extrawidth}
        \ifthenelse{\lengthtest{\firstwidth < \secondwidth}}
            {\addtolength{\turnstilewidth}{\secondwidth}}
            {\addtolength{\turnstilewidth}{\firstwidth}}}
        \ifthenelse{\lengthtest{\turnstilewidth < \leasturnstilewidth}}{\setlength{\turnstilewidth}{\leasturnstilewidth}}{}
    \setlength{\turnstileheight}{1.5ex}
    \sbox{\turnstilebox}
    {\raisebox{\lift}{\ensuremath{
        \makever{#2}{\dashthickness}{\turnstileheight}{\ddashthickness}
        \makehor{#3}{\dashthickness}{\turnstilewidth}{\ddashthickness}
        \hspace{-\turnstilewidth}
        \raisebox{\raisedown}
        {\makebox[\turnstilewidth]{\usebox{\first}}}
            \hspace{-\turnstilewidth}
            \raisebox{\raiseup}
            {\makebox[\turnstilewidth]{\usebox{\second}}}
        \makever{#6}{\dashthickness}{\turnstileheight}{\ddashthickness}}}}
        \mathrel{\usebox{\turnstilebox}}}

\newcommand{\newref}[4][]{
\ifthenelse{\equal{#1}{}}{\newtheorem{h#2}[hthm]{#4}}{\newtheorem{h#2}{#4}[#1]}
\expandafter\newcommand\csname r#2\endcsname[1]{#4~\ref{#2:##1}}
\newenvironmentx{#2}[2][1=,2=]{
\ifthenelse{\equal{##2}{}}{\begin{h#2}}{\begin{h#2}[##2]}
\ifthenelse{\equal{##1}{}}{}{\label{#2:##1}}
}{\end{h#2}}
}

\newref[section]{thm}{theorem}{Theorem}
\newref{lem}{lemma}{Lemma}
\newref{prop}{proposition}{Proposition}
\newref{cor}{corollary}{Corollary}

\theoremstyle{definition}
\newref{defn}{definition}{Definition}

\newcommand{\cat}[1]{\mathbf{#1}}
\renewcommand{\C}{\cat{C}}
\newcommand{\y}{\cat{y}}
\newcommand{\D}{\cat{D}}
\newcommand{\E}{\cat{E}}
\newcommand{\Set}{\cat{Set}}
\newcommand{\Grp}{\cat{Grp}}
\newcommand{\Mon}{\cat{Mon}}
\newcommand{\Ab}{\cat{Ab}}
\newcommand{\Ring}{\cat{Ring}}
\renewcommand{\Vec}{\cat{Vec}}
\newcommand{\Hask}{\cat{Hask}}
\newcommand{\Mat}{\cat{Mat}}
\newcommand{\Num}{\cat{Num}}
\newcommand{\red}{\Rightarrow}
\renewcommand{\ll}{\llbracket}
\newcommand{\rr}{\rrbracket}
\newcommand{\Mod}[1]{#1\text{-}\cat{Mod}}
\newcommand{\repl}{:=}

\newcommand{\fs}[1]{\mathrm{#1}}
\newcommand{\id}{\fs{id}}
\newcommand{\im}{\fs{Im}}
\newcommand{\Sub}{\fs{Sub}}

\newcommand{\pb}[1][dr]{\save*!/#1-1.2pc/#1:(-1,1)@^{|-}\restore}
\newcommand{\po}[1][dr]{\save*!/#1+1.2pc/#1:(1,-1)@^{|-}\restore}

\AtBeginSection[]
{
\begin{frame}[c,plain,noframenumbering]
\frametitle{План лекции}
\tableofcontents[currentsection]
\end{frame}
}

\makeatletter
\defbeamertemplate*{footline}{my theme}{
    \leavevmode
}
\makeatother

\begin{document}

\title{Теория категорий}
\subtitle{Категориальная логика}
\author{Валерий Исаев}
\maketitle

\section{Интерпретация}

\subsection{Подстановка}

\begin{frame}
\frametitle{Подстановка}
\begin{itemize}
\item Пусть $U$ и $V$ -- $\mathcal{S}$-индексированные множества переменных. Тогда подстановка из $U$ в $V$ (обозначается $\rho : U \to V$) -- это просто $V$-индексированное множество термов $\{ \rho(x) \in \fs{Term}(V)_s \}_{(x : s) \in V}$.
\item Подстановка $(x := a)$ -- это просто $\rho$ такая, что $\rho(x) = a$ и $\rho(y) = y$ для любого $y \neq x$.
\item Если $\rho : U \to V$ и $t \in \fs{Term}(V)_s$, то $t[\rho]$ -- это терм в $t \in \fs{Term}(U)_s$, который определяется рекурсивно:
\begin{itemize}
\item Если $t = x$, то $t[\rho] = \rho(x)$.
\item Если $t = f(t_1, \ldots t_k)$, то $t[\rho] = f(t_1[\rho], \ldots t_k[\rho])$.
\end{itemize}
\item Подстановка $\varphi[\rho]$ в формулы определяется рекурсией по построению $\varphi$.
\end{itemize}
\end{frame}

\begin{frame}
\frametitle{Интерпретация подстановки в термах}
Если $\rho : U \to V$, то мы определяем $\ll \rho \rr : \ll U \rr \to \ll V \rr$ как $\langle \rho(x) \rangle_{x \in V}$.
\begin{lem}
Если $t \in \fs{Term}(V)_s$ и $\rho : U \to V$, то $\ll t[\rho] \rr = \ll \rho \rr \circ \ll t \rr$.
\end{lem}
\begin{proof}
Индукцией по построению $t$.
\end{proof}
\end{frame}

\begin{frame}
\frametitle{Интерпретация подстановки в формулах}
\begin{lem}
Если $\varphi \in \fs{Form}_\Sigma(V)$ и $\rho : U \to V$, то существует (уникальный) морфизм $d_{\varphi[\rho]} \to d_\varphi$ такой, что следующий квадрат является пулбэком:
\[ \xymatrix{ d_{\varphi[\rho]} \ar[r] \ar[d]_{\ll \varphi[\rho] \rr} \pb & d_\varphi \ar[d]^{\ll \varphi \rr} \\
              \ll U \rr \ar[r]_-{\ll \rho \rr}        & \ll V \rr
            } \]
\end{lem}
\begin{itemize}
\item Доказывается индукцией по построению $\varphi$.
\item Сейчас мы проверим это утверждение только для атомарных формул, так как мы вводили только их пока.
\item Для других связок проверим позже.
\end{itemize}
\end{frame}

\begin{frame}
\frametitle{Интерпретация подстановки}
Если $\varphi = R(t_1, \ldots t_k)$, то мы получаем следующую диаграму:
\[ \xymatrix{ d_{\varphi[\rho]} \ar[r] \ar[d]_{\ll \varphi[\rho] \rr} \pb & d_\varphi \ar[d]^{\ll \varphi \rr} \ar[rr] \pb                                        & & d_R \ar[d]^{\ll R \rr} \\
              \ll U \rr \ar[r]_-{\ll \rho \rr}            & \ll V \rr \ar[rr]_-{\langle \ll t_1 \rr, \ldots \ll t_k \rr \rangle} & & \ll s_1 \rr \times \ldots \times \ll s_k \rr
            } \]
Правый квадрат -- это определение интерпретации $R(t_1, \ldots t_k)$.
Композиция нижних стрелок -- это $\langle \ll t_1[\rho] \rr, \ldots \ll t_k[\rho] \rr \rangle$.
По определению интерпретации $R(t_1[\rho], \ldots t_k[\rho])$ у на есть стрелка $d_{\varphi[\rho]} \to d_R$ такая, что внешний прямоугольник является пулбэком.
По свойствам пулбэков у нас есть стрелка $d_{\varphi[\rho]} \to d_\varphi$ такая, что левый квадрат является пулбэком.
\end{frame}

\begin{frame}
\frametitle{Интерпретация подстановки}
Если $\varphi = (t_1 = t_2)$, то в следующей диаграмме $\ll \varphi \rr$ по определению является уравнителем $\ll t_1 \rr, \ll t_2 \rr : \ll V \rr \to \ll s \rr$, а $\ll \varphi[\rho] \rr$ является уравнителем $\ll t_1 \rr \circ \ll \rho \rr$, $\ll t_2 \rr \circ \ll \rho \rr : \ll U \rr \to \ll s \rr$.
\[ \xymatrix{ d_{\varphi[\rho]} \ar[r] \ar[d]_{\ll \varphi[\rho] \rr} \pb & d_\varphi \ar[d]^{\ll \varphi \rr} \\
              \ll U \rr \ar[r]_-{\ll \rho \rr}        & \ll V \rr \ar@<-.5ex>[r]_{\ll t_2 \rr} \ar@<.5ex>[r]^{\ll t_1 \rr} & \ll s \rr
            } \]
По универсальному свойству уравнителей у нас есть стрелка $d_{\varphi[\rho]} \to d_\varphi$ и квадрат является пулбэком.
\end{frame}

\subsection{Корректность интерпретации}

\begin{frame}
\frametitle{Правила вывода}
Во всех наших логиках будут следующие правила вывода:
\begin{center}
\AxiomC{}
\UnaryInfC{$\varphi \sststile{}{V} \psi \land \varphi$}
\DisplayProof
\qquad
\AxiomC{$\varphi \sststile{}{V} \psi$}
\AxiomC{$\psi \sststile{}{V} \chi$}
\BinaryInfC{$\varphi \sststile{}{V} \chi$}
\DisplayProof
\end{center}

\begin{center}
\AxiomC{$\varphi \sststile{}{U} \psi$}
\UnaryInfC{$\varphi[\rho] \sststile{}{V} \chi[\rho]$}
\DisplayProof
\end{center}
С добавлением логических связок будут добавляться и правила вывода.
\end{frame}

\begin{frame}
\frametitle{Корректность интерпретации}
\begin{itemize}
\item Разумеется мы хотим, чтобы интерпретация уважала правила вывода.
\item Первые два правила очевидны: $\ll \varphi \rr$ является подобъектом самого себя и, если $\ll \varphi \rr$ является подобъектом $\ll \psi \rr$ и $\ll \psi \rr$ является подобъектом $\ll \chi \rr$, то $\ll \varphi \rr$ является подобъектом $\ll \chi \rr$.
\item Третье правило следует из леммы об интерпретации подстановки.
\end{itemize}
\end{frame}

\begin{frame}
\frametitle{Корректность $\top$}
\begin{itemize}
\item Правило вывода для $\top$:
\[ \varphi \sststile{}{} \top \]
\item Чтобы доказать, что эта аксиома всегда корректна, нужно проверить, что для любой подобъект $d_\varphi \hookrightarrow V$ является подобъектом $\ll \top \rr = id_V : V \hookrightarrow V$, что очевидно.
\end{itemize}
\end{frame}

\begin{frame}
\frametitle{Корректность $\land$}
\begin{itemize}
\item Правила вывода для $\land$:
\begin{center}
\AxiomC{$\varphi \sststile{}{} \psi$}
\AxiomC{$\varphi \sststile{}{} \chi$}
\BinaryInfC{$\varphi \sststile{}{} \psi \land \chi$}
\DisplayProof
\end{center}

\begin{center}
\AxiomC{$\varphi \sststile{}{} \psi \land \chi$}
\UnaryInfC{$\varphi \sststile{}{} \psi$}
\DisplayProof
\quad
\AxiomC{$\varphi \sststile{}{} \psi \land \chi$}
\UnaryInfC{$\varphi \sststile{}{} \chi$}
\DisplayProof
\end{center}
\item Эти правила уважаются, так как по определению пулбэков стрелка $\ll \varphi \rr \to \ll \psi \rr \cap \ll \chi \rr$ существует тогда и только тогда, когда существуют стрелки $\ll \varphi \rr \to \ll \psi \rr$ и $\ll \varphi \rr \to \ll \chi \rr$.
\end{itemize}
\end{frame}

\section{Регулярные теории}

\begin{frame}
\frametitle{Интерпретация $\exists$}
\begin{itemize}
\item Теории, в которых формулы состоят только из равенств, конъюнкций, $\top$ и $\exists$ называются \emph{регулярными}.
\item Мы не можем проинтерпретировать $\exists$ в произвольной конечно полной категории.
\item Категории, где можно это сделать, называются \emph{регулярными}.
\item Формальное определение будет дано ниже.
\end{itemize}
\end{frame}

\begin{frame}
\frametitle{Интерпретация $\exists$ в $\Set$}
\begin{itemize}
\item Пусть $\ll \varphi(x, x_1, \ldots x_n) \rr : d_\varphi \hookrightarrow \ll s \rr \times \ll s_1 \rr \times \ldots \times \ll s_n \rr$.
\item Как проинтерпретировать $\exists (x : s) (\varphi(x, x_1 \ldots x_n))$?
\item Если рассмотреть $\pi_{1, \ldots n} \circ \ll \varphi \rr : d_\varphi \to \ll s_1 \rr \times \ldots \times \ll s_n \rr$, то это почти дает нам интерпретацию $\exists (x : s) (\varphi(x, x_1, \ldots x_n))$,
так как прообраз некоторого элемента $(a_1, \ldots a_n)$ населен тогда и только тогда, когда существует $a \in \ll s \rr$, такой что $(a, a_1, \ldots a_n) \in \ll \varphi \rr$.
\item Единственная проблема заключается в том, что $\pi_{1, \ldots n} \circ \ll \varphi \rr$ не является мономорфизмом.
\item Мы можем решить эту проблему, определив интерпретацию $\exists (x : s) (\varphi(x, x_1, \ldots x_n))$ как образ $\pi_{1, \ldots n} \circ \ll \varphi \rr$.
\end{itemize}
\end{frame}

\begin{frame}
\frametitle{Образ морфизма}
\begin{itemize}
\item Мы можем обобщить понятие образа функции на произвольную категорию.
\item \emph{Образ} морфизма $f : A \to B$ -- это наименьший мономорфизм $\im\,f \hookrightarrow B$, через который $f$ факторизуется.
\item Другими словами, существует стрелка $A \to \im\,f$, такая что для любых стрелок $g : A \to C$ и $h : C \hookrightarrow B$ если $h \circ g = f$,
то $\im\,f$ является подобъектом $C$:
\[ \xymatrix{                               & \im\,f \ar[dr] \ar@{-->}[dd] &   \\
              A \ar[ur] \ar[dr]_g \ar[rr]_f &                              & B \\
                                            & C \ar[ur]_h
            } \]
\end{itemize}
\end{frame}

\begin{frame}
\frametitle{Интерпретация существования}
\begin{itemize}
\item В произвольной категории образ может не существовать, но он уникален, если существует.
\item Если предположить, что в категории существуют образы, то можно попробовать проинтерпретировать существование так же как и в $\Set$.
\item Если $\ll \varphi(x, x_1, \ldots x_n) \rr : d_\varphi \hookrightarrow \ll s \rr \times \ll s_1 \rr \times \ldots \times \ll s_n \rr$,
то мы определяем $\ll \exists (x : s) \varphi \rr$ как образ $\pi_{1, \ldots n} \circ \ll \varphi \rr$.
\end{itemize}
\end{frame}

\begin{frame}
\frametitle{Интерпретация подстановки}
\begin{itemize}
\item Так как правила вывода для существования используют подстановку, нам нужно знать как она интерпретируется, чтобы проверить эти правила.
\item Утверждене: если $\varphi$ -- формула со свободными переменными в $\{ x_1 : s_1, \ldots x_n : s_n \}$ и $t_1$, \ldots $t_n$ -- термы сортов $s_1$, \ldots $s_n$,
то $\ll \varphi[x_1 \repl t_1, \ldots x_n \repl t_n] \rr$ является пулбэком $d_\varphi$:
\[ \xymatrix{ d_{\varphi[x_1 \repl t_1, \ldots x_n \repl t_n]} \ar[r] \ar[d]_{\ll \varphi[x_1 \repl t_1, \ldots x_n \repl t_n] \rr} \pb & d_\varphi \ar[d]^{\ll \varphi \rr} \\
              V \ar[r]_-{\langle t_1, \ldots t_n \rangle}                                                                               & \ll s_1 \rr \times \ldots \times \ll s_n \rr
            } \]
\end{itemize}
\end{frame}

\begin{frame}
\frametitle{Интерпретация подстановки}
\begin{itemize}
\item Доказывать это утверждение мы будем индукцией по $\varphi$.
\item Для $\top$ это очевидно, так как пулбэк тождественного морфизма -- тождественный морфизм.
\item Проверим для равенства. Пусть $\ll t \rr, \ll t' \rr : \ll s_1 \rr \times \ldots \times \ll s_n \rr \to \ll s' \rr$ и $\ll t_i \rr : V \to \ll s_i \rr$.
Тогда нужно показать, что морфизм $E \to V$ в диаргамме ниже является уровнителем $\ll t \rr \circ \langle \ll t_1 \rr, \ldots \ll t_n \rr \rangle$ и $\ll t' \rr \circ \langle \ll t_1 \rr, \ldots \ll t_n \rr \rangle$,
что легко сделать, используя универсальное свойство пулбэков.
\[ \xymatrix{ E \ar[rr] \ar[d] \pb                                          & & d_{t = t'} \ar[d]^{\ll t = t' \rr} \\
              V \ar[rr]_-{\langle \ll t_1 \rr, \ldots \ll t_n \rr \rangle}  & & \ll s_1 \rr \times \ldots \times \ll s_n \rr \ar@<-.5ex>[r]_-{\ll t' \rr} \ar@<.5ex>[r]^-{\ll t \rr} & \ll s' \rr
            } \]
\end{itemize}
\end{frame}

\begin{frame}
\frametitle{Интерпретация подстановки в $\exists$}
\begin{itemize}
\item Даже если в категории существуют образы всех морфизмов, они могут не коммутировать с пулбэками.
\item Категория называется \emph{регулярной}, если у всех морфизмов существуют образы, и они стабильны относительно пулбэков.
\item Таким образом, в любой регулярной категории подстановка действительно интерпретируется как пулбэк.
\end{itemize}
\end{frame}

\begin{frame}
\frametitle{Корректность интерпретации $\exists$}
\begin{itemize}
\item Правила вывода для $\exists$:
\begin{center}
\AxiomC{$\exists (x : s) \varphi \sststile{}{} \psi$}
\UnaryInfC{$\varphi \sststile{}{} \psi$}
\DisplayProof
\quad
\AxiomC{$\varphi \sststile{}{} \psi$}
\UnaryInfC{$\exists (x : s) \varphi \sststile{}{} \psi$}
\DisplayProof
\end{center}
\item Обратите внимание, что $\psi$ определен в контексте $x_1, \ldots x_n$, но используется также и в контексте $x_1, \ldots x_n, x$.
По лемме об интерпретации подстановки $\psi$ во втором контексте интерпретируется как пулбэк.
\item Используя этот факт, легко показать, что данные правила корректны.
\end{itemize}
\end{frame}

\section{Когерентные теории}

\begin{frame}
\frametitle{Определение}
\begin{itemize}
\item Категория называется \emph{когерентной}, если она регулярна, для любого объекта $A$ в порядке подобъектов $\Sub(A)$ существуют все конечные копроизведения,
и для любого морфизма $f : A \to B$ функтор $f^* : \Sub(B) \to \Sub(A)$ сохраняет их.
\item Эта дополнительная структура -- это в точности то, что необходимо для интерпретации ложного утверждения и дизъюнкций.
\end{itemize}
\end{frame}

\begin{frame}
\frametitle{Дистрибутивность пересечений}
\begin{prop}
В когерентной категории пересечении дистрибутивно над объединением подобъектов: $A \cap (B \cup C) \simeq (A \cap B) \cup (A \cap C)$ и $A \cap 0 = 0$.
\end{prop}
\begin{proof}
Функтор $A \cap - : \Sub(X) \to \Sub(X)$ можно определить как композицию
\[ \Sub(X) \xrightarrow{f^*} \Sub(A) \xrightarrow{\exists_f} \Sub(X) \]
где $f : A \to X$ и $\exists_f$ -- левый сопряженный к $f^*$.
Функтор $f^*$ сохраняет копроизведения по предположению когерентности, а $\exists_f$ сохраняет копределы как левый сопряженный.
\end{proof}
\end{frame}

\begin{frame}
\frametitle{Начальный объект}
\only<1>{
\begin{prop}
В когерентной категории существует строгий начальный объект.
\end{prop}
}
\begin{proof}
\only<1>{
Определим 0 как наименьший подобъект 1.
Заметим, что $\pi_1 : X \times 0 \hookrightarrow X$ является наименьшим подобъектом $X$.
Если у нас есть стрелка $X \to 0$, то $\pi_1$ является изоморфизмом.
Другими словами, он является и наибольшим подобъектом $X$.
Следовательно, у любого такого $X$ ровно один подобъект -- он сам.
}
\only<2>{
Докажем, что если есть морфизм $A \to 0$, то $A$ является подобъектом 1.
Дествительно, если у нас есть пара стрелок $f,g : B \to A$,
то так как у нас есть стрелка $B \to 0$, то уравнитель $f$ и $g$ является изоморфизмом, то есть $f$ и $g$ равны.
Следовательно $X \times 0$ изоморфен 0, то есть 0 -- строгий.

Докажем, что 0 -- начальный.
Так как у нас есть стрелка из $X \times 0$ в 0, то $X \times 0 \simeq 0$, а значит у нас есть стрелка из $0$ в $X$.
Если у нас есть стрелки $f,g : 0 \to X$, то их уравнитель является подобъектом 0, а значит изоморфизмом, то есть $f$ и $g$ равны.
}
\end{proof}
\end{frame}

\section{Импликация и $\forall$}

\begin{frame}
\frametitle{Квантор всеобщности}
\begin{itemize}
\item Правила вывода для $\forall$ дуальны правилам для $\exists$:
\begin{center}
\AxiomC{$\varphi \sststile{}{} \forall (x : s) \psi$}
\UnaryInfC{$\varphi \sststile{}{} \psi$}
\DisplayProof
\quad
\AxiomC{$\varphi \sststile{}{} \psi$}
\UnaryInfC{$\varphi \sststile{}{} \forall (x : s) \psi$}
\DisplayProof
\end{center}
\item То есть у нас есть биекция между стрелками $\pi_1^*(\ll \varphi \rr) \to \ll \psi \rr$ в $\Sub(X \times \ll s \rr)$ и $\ll \varphi \rr \to \ll \forall (x : s) \psi \rr$ в $\Sub(X)$, где $\pi_1 : X \times \ll s \rr \to X$.
\item Таким образом, $\ll \forall (x : s) \psi \rr$ можно определить как $\forall_{\pi_1}(\ll \psi \rr)$, где $\forall_{\pi_1} : \Sub(X \times \ll s \rr) \to \Sub(X)$ -- правый сопряженный к $\pi_1^* : \Sub(X) \to \Sub(X \times \ll s \rr)$.
\end{itemize}
\end{frame}

\begin{frame}
\frametitle{Гейтинговы категории}
\begin{itemize}
\item Категория называется \emph{гейтинговой}, если она регулярна, у любого объекта существует минимальный подобъект и объединения подобъектов,
и для любого морфизма $f : X \to Y$ существует правый сопряженный функтор $\forall_f : \Sub(Y) \to \Sub(X)$ к функтору $f^* : \Sub(Y) \to \Sub(X)$.
\item Так как гейтингова категория регулярна, то у функтора $f^*$ есть и левый сопряженный. Таким образом, мы получаем цепочку сопряженных функторов:
\[ \exists_f \dashv f^* \dashv \forall_f \]
\item Мы не требуем, чтобы $f^*$ сохранял наименьший подобъект и объединения, так как это следует из того, что он левый сопряженный.
\end{itemize}
\end{frame}

\begin{frame}
\frametitle{$\forall$ коммутирует с подстановкой}
Чтобы доказать, что $\forall$ коммутирует с подстановкой, нам нужно доказать, что для любого пулбэка слева квадрат функторов справа коммутирует с точностью до изоморфизма.
\[ \xymatrix{ P \ar[r]^h \ar[d]_k \pb & A \ar[d]^f \\
              B \ar[r]_g              & C
            } \qquad
   \xymatrix{ \Sub(A) \ar[r]^{h^*} \ar[d]_{\forall_f} & \Sub(P) \ar[d]^{\forall_k} \\
              \Sub(C) \ar[r]_{g^*}                    & \Sub(B)
            } \]
\end{frame}

\begin{frame}
\frametitle{$\forall$ коммутирует с подстановкой}
\begin{itemize}
\item Функторы в правом квадрате являются правыми сопряженными к следующим функторам:
\[ \xymatrix{ \Sub(A)              & \Sub(P) \ar[l]_{\exists_h} \\
              \Sub(C) \ar[u]^{f^*} & \Sub(B) \ar[l]^{\exists_g} \ar[u]_{k^*} 
            } \]
\item Так как этот квадрат коммутирует по регулярности, то квадрат на предыдущем слайде коммутирует по уникальности правых сопряженных функторов.
\end{itemize}
\end{frame}

\begin{frame}
\frametitle{Импликация}
\begin{itemize}
\item Интерпретация импликации определяется так же как интерпретация экспонент.
\item То есть $\ll \varphi \to (-) \rr : \Sub(X) \to \Sub(X)$ -- правый сопряженный к $\ll \varphi \land - \rr = \ll \varphi \rr \cap -$.
\item Так как $A \cap -$ является композицией
\[ \Sub(X) \xrightarrow{f^*} \Sub(A) \xrightarrow{\exists_f} \Sub(X) \]
где $f : A \hookrightarrow X$, то правый сопряженный к нему существует в любой гейтинговой категориии.
\item Таким образом, мы можем проинтерпретировать $\varphi \to \psi$ как $\forall_{\ll \varphi \rr} (\ll \varphi \rr^*(\psi))$.
\item Так как $\forall_{\ll \varphi \rr}$ и $\ll \varphi \rr^*$ коммутируют с подстановкой, то и интерпретация импликации с ней коммутирует.
\end{itemize}
\end{frame}

\end{document}
