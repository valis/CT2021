\documentclass[draft]{article}
\usepackage[russian]{babel}
\usepackage[utf8]{inputenc}
\usepackage{cmap}
\usepackage{amsfonts}
\usepackage[all]{xy}

\newcommand{\cat}[1]{\mathbf{#1}}
\renewcommand{\C}{\cat{C}}
\newcommand{\Set}{\cat{Set}}
\newcommand{\Grp}{\cat{Grp}}
\newcommand{\Ab}{\cat{Ab}}
\newcommand{\Vec}{\cat{Vec}}
\newcommand{\Hask}{\cat{Hask}}
\newcommand{\Mat}{\cat{Mat}}
\newcommand{\Num}{\cat{Num}}

\begin{document}

\title{Задания}
\maketitle

\begin{enumerate}

\item Опишите в категории (пред)порядка следующие конструкции:
\begin{enumerate}
\item Терминальные объекты.
\item Произведения объектов.
\end{enumerate}

\item Пусть в категории $\C$ существует терминальный объект 1.
Докажите, что для любого объекта $A$ в $\C$ существует произведение $A \times 1$.

\item Докажите, что любой морфизм из терминального объекта является мономорфизмом.

\item Пусть в категории $\C$ существует терминальный объект 1 и некоторый морфизм $1 \to B$.
Докажите, что любая проекция $\pi_1 : A \times B \to A$ является эпиморфизмом.

\item Докажите, что в $\Ab$ существуют все произведения.

\item Докажите, что два определения уравнителей, приводившихся в лекции, эквивалентны.

\item Докажите, что уравнитель пары стрелок $f,g : A \to B$ уникален с точностью до изоморфизма.
То есть, если $e_1 : E_1 \to A$ и $e_2 : E_2 \to A$ -- два уравнителя $f$ и $g$, то существует уникальный изоморфизм $i : E_1 \to E_2$ такой, что $e_2 \circ i = e_1$.

\item Морфизм $h : B \to B$ называется \emph{идемпотентным}, если $h \circ h = h$.
Докажите следующие факты:
\begin{enumerate}
\item Если $f : A \to B$ и $g : B \to A$ -- такие, что $g \circ f = id_A$, то $h = f \circ g$ является идемпотентным.
\item Если в категории есть уравнители, то обратное верно.
Конкретно, для любого идемпотентного морфизма $h : B \to B$ существуют $f : A \to B$ и $g : B \to A$ такие, что $g \circ f = id_A$ и $f \circ g = h$.
\end{enumerate}

\item Докажите, что любой расщепленный мономорфизм регулярен.

\item Мономорфизм $f : A \to B$ называется \emph{сильным}, если для любой коммутативного квадрата, где $e : C \to D$ является эпиморфизмом,
\[ \xymatrix{ C \ar[r] \ar[d]_e      & A \ar[d]^f \\
              D \ar[r] \ar@{-->}[ur] & B
            } \]
существует стрелка $D \to A$ такая, что диаграмма выше коммутирует.

Докажите, что любой регулярный мономорфизм силен.

\item Мономорфизм $f : A \to B$ называется \emph{экстремальным}, если для любого эпиморфизма $e : A \to C$ и любого морфизма $g : C \to B$ таких, что $g \circ e = f$, верно, что $e$ -- изоморфизм.

Докажите, что любой сильный мономорфизм экстремален.

\item Докажите, что если в категории все мономорфизмы регулярны, то она сбалансирована. Можно ли усилить это утверждение?

\item Докажите, что в $\Set$ все мономорфизмы регулярны.

\item Докажите, что в $\Ab$ все мономорфизмы регулярны.

\end{enumerate}

Бонусные задания:

\begin{enumerate}

\item Докажите, что если в категории $\C_M$ существуют бинарные произведения и моноид $M$ нетривиален, то он бесконечен.

\item Докажите, что если в категории $\C_M$ существуют бинарные произведения и моноид $M$ нетривиален, то для любого натурального $n > 1$ существует $x \in M$ такой, что $x \neq 1$ и $x^n = 1$.

\item Приведите пример нетривиального моноида $M$ такого, что в категории $\C_M$ существует бинарные произведения.

\end{enumerate}

\end{document}
