\documentclass[draft]{article}
\usepackage[russian]{babel}
\usepackage[utf8]{inputenc}
\usepackage{cmap}
\usepackage{amsfonts}
\usepackage{amssymb}
\usepackage{amsmath}
\usepackage[all]{xy}
\usepackage{stmaryrd}
\usepackage{bussproofs}

\newcommand{\cat}[1]{\mathbf{#1}}
\renewcommand{\C}{\cat{C}}
\newcommand{\D}{\cat{D}}
\newcommand{\y}{\cat{y}}
\newcommand{\Set}{\cat{Set}}
\newcommand{\Grp}{\cat{Grp}}
\newcommand{\Ab}{\cat{Ab}}
\newcommand{\Hask}{\cat{Hask}}
\newcommand{\Mat}{\cat{Mat}}
\newcommand{\Num}{\cat{Num}}
\newcommand{\red}{\Rightarrow}
\renewcommand{\ll}{\llbracket}
\newcommand{\rr}{\rrbracket}

\newcommand{\pb}[1][dr]{\save*!/#1-1.2pc/#1:(-1,1)@^{|-}\restore}
\newcommand{\po}[1][dr]{\save*!/#1+1.2pc/#1:(1,-1)@^{|-}\restore}

\newcommand{\im}{\mathrm{Im}}
\newcommand{\bool}{\mathrm{Bool}}
\newcommand{\true}{\mathrm{true}}
\newcommand{\false}{\mathrm{false}}
\newcommand{\andb}{\mathrm{and}}
\newcommand{\orb}{\mathrm{or}}
\newcommand{\inj}{\mathrm{inj}}

\newcommand{\ev}{\mathrm{ev}}
\newcommand{\zero}{\mathrm{zero}}
\newcommand{\suc}{\mathrm{suc}}
\newcommand{\rec}{\mathrm{rec}}

\newenvironment{tolerant}[1]{\par\tolerance=#1\relax}{\par}

\begin{document}

\title{Задания}
\maketitle

\begin{enumerate}

\item Пусть $\C$ -- конечно полная категория.
Тогда для любого морфизма $f : A \to B$ можно определть функтор $f^* : \mathrm{Sub}(B) \to \mathrm{Sub}(A)$, где $\mathrm{Sub}(X)$ -- полная подкатегория $\C/X$, объекты которой -- это стрелки $Y \to X$, являющиеся мономорфизмами.
Докажите, что следующие утверждения эквивалентны:
\begin{enumerate}
\item У любого морфизма $f : A \to B$ существует образ $\mathrm{im}\,f \hookrightarrow B$.
\item Для любого морфизма $f : A \to B$ у функтора $f^*$ есть левый сопряженный функтор $\exists_f : \mathrm{Sub}(A) \to \mathrm{Sub}(B)$.
\end{enumerate}

\item Пусть $\C$ -- конечно полная категория. На лекции доказывалось, что если в $\C$ можно проинтерпретировать $\bot$ (то есть у любого объекта существует наименьший подобъект и наименьшие подобъекты стабильны относительно пулбэков), то в $\C$ есть строгий начальный объект.
Докажите обратное утверждение, то есть что, если в $\C$ есть строгий начальный объект, то в ней можно проинтерпретировать $\bot$.

\item Категория $\C$ называется локально декартово замкнутой, если в ней существуют конечные пределы и $\C/A$ является декартово замкнутой для любого объекта $A$.
Докажите, что если в локально декартово замкнутой регулярной категории существуют все конечные суммы, то она гейтингова.

\end{enumerate}

\end{document}
