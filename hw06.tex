\documentclass[draft]{article}
\usepackage[russian]{babel}
\usepackage[utf8]{inputenc}
\usepackage{cmap}
\usepackage{amsfonts}
\usepackage{amssymb}
\usepackage{amsmath}
\usepackage[all]{xy}

\newcommand{\cat}[1]{\mathbf{#1}}
\renewcommand{\C}{\cat{C}}
\newcommand{\D}{\cat{D}}
\newcommand{\Set}{\cat{Set}}
\newcommand{\FinSet}{\cat{FinSet}}
\newcommand{\Grp}{\cat{Grp}}
\newcommand{\CMon}{\cat{CMon}}
\newcommand{\Ab}{\cat{Ab}}
\newcommand{\Mat}{\cat{Mat}}
\newcommand{\Num}{\cat{Num}}
\newcommand{\fs}[1]{\mathrm{#1}}
\newcommand{\Ob}{\fs{Ob}}
\newcommand{\Hom}{\fs{Hom}}

\newenvironment{tolerant}[1]{\par\tolerance=#1\relax}{\par}

\begin{document}

\title{Задания}
\maketitle

\begin{enumerate}

\item Пусть $\C$ -- категория предпорядка, а $\D$ -- нет.
\begin{enumerate}
\item Могут ли $\C$ и $\D$ быть изоморфны?
\item Могут ли $\C$ и $\D$ быть эквивалентны?
\end{enumerate}

\item Пусть $\C$ -- категория с одним объектом, а $\D$ -- нет.
\begin{enumerate}
\item Могут ли $\C$ и $\D$ быть изоморфны?
\item Могут ли $\C$ и $\D$ быть эквивалентны?
\end{enumerate}

\item Пусть $\C$ -- дискретная категория, а $\D$ -- нет.
\begin{enumerate}
\item Могут ли $\C$ и $\D$ быть изоморфны?
\item Могут ли $\C$ и $\D$ быть эквивалентны?
\end{enumerate}

\item Пусть $\C$ -- группоид, а $\D$ -- нет.
\begin{enumerate}
\item Могут ли $\C$ и $\D$ быть изоморфны?
\item Могут ли $\C$ и $\D$ быть эквивалентны?
\end{enumerate}

\item Докажите, что $\Num$ эквивалентна $\FinSet$. Изоморфны ли эти категории?

\item Докажите, что $\Mat$ эквивалентна $\Mat^{op}$. Изоморфны ли эти категории?

\item Докажите, что $\FinSet$ не эквивалентна $\Set$.

\item Пусть $F, G : \C \to \D$ -- пара функторов.
Естественное преобразование $\alpha : F \to G$ называется \emph{естественным изоморфизмом}, если для любого объекта $X$ в $\C$ морфизм $\alpha_X : F(X) \to G(X)$ является изоморфизмом.

Докажите, что $\alpha : F \to G$ -- естественный изоморфизм тогда и только тогда, когда $\alpha$ -- изоморфизм в категории $\D^\C$.

\item Пусть $\C$ -- декартова категория. Докажите, что функтор $- \times 1$ изоморфен тождественному функтору в $\C^\C$.

\item Пусть $\pmb{\rightrightarrows}$ -- категория, состоящая из двух объектов $\{ v, e \}$ и четырех морфизмов $\{ id_v : v \to v, id_e : e \to e, d : v \to e, c : v \to e \}$.
Докажите, что категории $\cat{Graph}$ (эта категория определяется в предыдущем ДЗ) и $\Set^{\pmb{\rightrightarrows}^{op}}$ эквивалентны.
Изоморфны ли эти категории?

\item Пусть $\D$ -- рефлективная подкатегория $\C$.
\begin{enumerate}
\item Докажите, что рефлектор $\Ob(\C) \to \Ob(\D)$ является фнуктором $R : \C \to \D$.
\item Докажите, что $\eta$ является естественным преобразованием между $\fs{Id}_\C$ и $i \circ R$, где $i : \D \to \C$ -- функтор вложения.
\end{enumerate}

\item Пусть $F : \CMon \to \Ab$ -- рефлектор вложения $i : \Ab \to \CMon$.
\begin{enumerate}
\item \tolerant{500}{ Приведите пример конечного нетривиального коммутативного моноида $X$, такого что $|F(X)| = |X|$. }
\item Приведите пример конечного коммутативного моноида $X$, такого что $|F(X)| < |X|$.
\item Приведите пример коммутативного моноида $X$, такого что $\eta_X : X \to i(F(X))$ -- не сюръективна.
\item Докажите, что для любого конечного коммутативного моноида $X$ функция $\eta_X : X \to i(F(X))$ является сюръективной. В частности $|F(X)| \leq |X|$.
\end{enumerate}

\end{enumerate}

\end{document}
