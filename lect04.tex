\documentclass{beamer}

\usepackage[english,russian]{babel}
\usepackage[utf8]{inputenc}
\usepackage[all]{xy}
\usepackage{ifthen}
\usepackage{xargs}
\usepackage{bussproofs}

\usetheme{Szeged}
% \usetheme{Montpellier}
% \usetheme{Malmoe}
% \usetheme{Berkeley}
% \usetheme{Hannover}
\usecolortheme{beaver}

\newcommand{\newref}[4][]{
\ifthenelse{\equal{#1}{}}{\newtheorem{h#2}[hthm]{#4}}{\newtheorem{h#2}{#4}[#1]}
\expandafter\newcommand\csname r#2\endcsname[1]{#4~\ref{#2:##1}}
\newenvironmentx{#2}[2][1=,2=]{
\ifthenelse{\equal{##2}{}}{\begin{h#2}}{\begin{h#2}[##2]}
\ifthenelse{\equal{##1}{}}{}{\label{#2:##1}}
}{\end{h#2}}
}

\newref[section]{thm}{theorem}{Theorem}
\newref{lem}{lemma}{Lemma}
\newref{prop}{proposition}{Proposition}
\newref{cor}{corollary}{Corollary}

\theoremstyle{definition}
\newref{defn}{definition}{Definition}

\newcommand{\cat}[1]{\mathbf{#1}}
\renewcommand{\C}{\cat{C}}
\newcommand{\Set}{\cat{Set}}
\newcommand{\Grp}{\cat{Grp}}
\newcommand{\Ab}{\cat{Ab}}
\newcommand{\Ring}{\cat{Ring}}
\renewcommand{\Vec}{\cat{Vec}}
\newcommand{\Mat}{\cat{Mat}}
\newcommand{\Num}{\cat{Num}}
\newcommand{\fs}[1]{\mathrm{#1}}
\newcommand{\Hom}{\fs{Hom}}

\newcommand{\im}{\mathrm{Im}}
\newcommand{\bool}{\mathrm{Bool}}
\newcommand{\true}{\mathrm{true}}
\newcommand{\false}{\mathrm{false}}

\newcommand{\ev}{\mathrm{ev}}
\newcommand{\zero}{\mathrm{zero}}
\newcommand{\suc}{\mathrm{suc}}
\newcommand{\fst}{\mathrm{fst}}
\newcommand{\snd}{\mathrm{snd}}
\newcommand{\unit}{\mathrm{unit}}

\newcommand{\repl}{:=}
\renewcommand{\ll}{\llbracket}
\newcommand{\rr}{\rrbracket}

\newcommand{\pb}[1][dr]{\save*!/#1-1.2pc/#1:(-1,1)@^{|-}\restore}
\newcommand{\po}[1][dr]{\save*!/#1+1.2pc/#1:(1,-1)@^{|-}\restore}

\AtBeginSection[]
{
\begin{frame}[c,plain,noframenumbering]
\frametitle{План лекции}
\tableofcontents[currentsection]
\end{frame}
}

\makeatletter
\defbeamertemplate*{footline}{my theme}{
    \leavevmode
}
\makeatother

\begin{document}

\title{Теория категорий}
\subtitle{Декартово замкнутые категории}
\author{Валерий Исаев}
\maketitle

\section{Булевские объекты}

\begin{frame}
\frametitle{Копроизведение $1 \amalg 1$}
\begin{itemize}
\item В $\Set$ множество $\bool$ можно определить как копроизведение множеств $\{ \true \}$ и $\{ \false \}$, каждое из которых является терминальным.
\item Копроизведение $1 \amalg 1$ обычно обозначается как $2$.
\item Можно было бы в произвольной категории определить объект $\bool$ как копроизведение $1 \amalg 1$.
\item Но это недостаточно сильное определение. Мы не сможем никаких функций над ним определить.
\end{itemize}
\end{frame}

\begin{frame}
\frametitle{Булевский объект}
\begin{itemize}
\item Пусть в $\C$ существуют все конечные произведения.
\item Тогда \emph{булевский объект} в $\C$ -- это объект $\bool$ вместе с парой морфизмов $\true, \false : 1 \to \bool$, удовлетворяющий следующему условию.
\item Для любых $f, g : A \to B$ существует уникальная стрелка $h : \bool \times A \to B$, такая что
\[ \xymatrix{ A \ar[rr]^-{\langle \true \circ !_A, id_A \rangle} \ar[drr]_{f} & & \bool \times A \ar[d]^h \\
                                                                             & & B
            }
\qquad \xymatrix{ A \ar[rr]^-{\langle \false \circ !_A, id_A \rangle} \ar[drr]_{g} & & \bool \times A \ar[d]^h \\
                                                                              & & B
            }
\]
\end{itemize}
\end{frame}

\begin{frame}
\frametitle{Булевский объект и 2}
\begin{itemize}
\item Любой булевский объект является 2.
\item Действительно, если в определении булевского объекта в качестве $A$ взять 1, то мы получим в точности универсальное свойство $1 \amalg 1$.
\item Следовательно булевский объект уникален с точностью до изоморфизма.
\item Но не любой объект, являющийся 2, является булевским.
\item Действительно, в категории групп 2 изоморфен 1.
\item Но булевский объект изоморфен 1 только в категориях предпорядка.
\end{itemize}
\end{frame}

\begin{frame}
\frametitle{if}
\begin{itemize}
\item Мы можем сконструировать морфизм $if : \bool \times (C \times C) \to C$, удовлетворяющий
\[ \xymatrix{ C \times C \ar[d]_{\langle \true \circ !, id \rangle} \ar[rd]^{\pi_1} \\
              \bool \times (C \times C) \ar[r]_-{if} & C
            }
\quad \xymatrix{ C \times C \ar[d]_{\langle \false \circ !, id \rangle} \ar[rd]^{\pi_2} \\
              \bool \times (C \times C) \ar[r]_-{if} & C
            } \]
\item Действительно, в определении $\bool$ возьмем $A = C \times C$, $B = C$, $f = \pi_1$ и $g = \pi_2$.
\item Тогда существует уникальная стрелка $\bool \times (C \times C) \to C$, удовлетворяющая условиям выше.
\end{itemize}
\end{frame}

\section{Декартово замкнутые категории}

\begin{frame}
\frametitle{Мотивация}
\begin{itemize}
\item Очередная конструкция, которую мы хотим обобщить, -- это множество/тип функций.
\item Эта конструкция называется по разному: экспонента, внутренний $\Hom$.
\item Пусть $A$ и $B$ -- объекты декартовой категории $\C$. Тогда экспонента обозначаются либо $B^A$, либо $[A,B]$.
\item Какие операции должны быть определены для $B^A$.
\item Как минимум мы должны иметь аппликацию, которая обычно обозначается $\ev$ и является следующим морфизмом:
\[ \ev : B^A \times A \to B \]
\item Морфизм $\ev$ позволяет нам ``вычислять'' элементы $B^A$.
\end{itemize}
\end{frame}

\begin{frame}
\frametitle{Элементы объекта (a side note)}
\begin{itemize}
\item В категории $\Set$ элементы множества $X$ соответствуют морфизмам из термнального объекта в $X$.
\item В произвольной категории (с терминальным объектом) мы можем определить элемент объекта таким же образом.
\item Но это не очень полезное определение, так как в произвольной категории объект не определяется своими элементами.
\item Например, в категории графов морфизмы из терминального графа в граф $X$ соответствуют петлям $X$.
\item Мы можем определить \emph{обощенный элемент} объекта $X$ как морфизм из произвольного объекта $\Gamma$ в $X$.
\item В категории графов вершины и ребра графа $X$ являются его обобщенными элементами (конечно, существует и много других обобщенных элементов этого графа).
\end{itemize}
\end{frame}

\begin{frame}
\frametitle{Определение}
\begin{itemize}
\item Благодаря морфизму $\ev$, мы можем думать об элементах $B^A$ как о морфизмах $A \to B$.
Мы еще должны сказать, что $B^A$ содержит \emph{все} такие морфизмы.
\item То есть мы должны сказать чему соответствуют обобщенные элементы $B^A$.
Ясно, что у нас должна быть биекция между обобщенными элементами $\Gamma \to B^A$ и морфизмами $\Gamma \times A \to B$.
\item Имея морфизм $f : \Gamma \to B^A$, мы можем построить его каррирование следующим образом:
\[ \Gamma \times A \xrightarrow{f \times id_A} B^A \times A \xrightarrow{\ev} B\]
\item Объект $B^A$ вместе с морфизмом $\ev : B^A \times A \to B$ называется \emph{экспонентой} $A$ и $B$,
если для любого $g : \Gamma \times A \to B$ существует уникальный $f : \Gamma \to B^A$ такой, что композиция стрелок в диаграмме выше равна $g$.
\end{itemize}
\end{frame}

\begin{frame}
\frametitle{Примеры}
\begin{itemize}
\item Категория называется \emph{декартово замкнутой}, если она декартова и для любых ее объектов $A$ и $B$ существует их экспонента $B^A$.
\item $\Set$ -- декартово замкнута. Действительно, $B^A$ -- это просто множество функций из $A$ в $B$.
\item $\Lambda$ -- декартово замкнута. Действительно, $B^A$ -- это просто тип функций из $A$ в $B$.
\item Все алгебраические категории, которые мы рассматривали, не являются декартово замкнутыми ($\Grp$, $\Vec$, $\Ring$, и т.д.).
\item Категория графов -- декартово замнута.
\end{itemize}
\end{frame}

\begin{frame}
\frametitle{Объект натуральных чисел}
\begin{defn}
\emph{Объект натуральных чисел} в декартово замкнутой категории -- это объект $\mathbb{N}$ вместе с парой морфизмов $\zero : 1 \to \mathbb{N}$ и $\suc : \mathbb{N} \to \mathbb{N}$, удовлетворяющие условию, что
для любых других морфизмов $z : 1 \to X$ и $s : X \to X$ существует уникальная стрелка $h$, такая что диаграмма ниже коммутирует.
\[ \xymatrix{ 1 \ar[r]^\zero \ar[rd]_z  & \mathbb{N} \ar[r]^\suc \ar@{-->}[d]^h & \mathbb{N} \ar@{-->}[d]^h \\
                                        & X \ar[r]_s                            & X
            } \]
\end{defn}
\end{frame}

\begin{frame}
\frametitle{Свойства}
\begin{itemize}
\item Объект натуральных чисел уникален с точностью до изоморфизма.
\item В любой декартово замкнутой категории с объектом натуральных чисел можно определить все примитивно рекурсивные функции.
\item Морфизм $\suc : \mathbb{N} \to \mathbb{N}$ является расщепленным мономорфизмом.
\end{itemize}
\end{frame}

\end{document}
